\RequirePackage[l2tabu, orthodox]{nag}

\documentclass{article}

\usepackage[margin=1.9cm, letterpaper]{geometry}
\usepackage[utf8]{inputenc}
\usepackage[a-1b]{pdfx}
\usepackage{indentfirst}
\usepackage{float}
\usepackage{subcaption}
\usepackage{mathtools}
\usepackage{amssymb}
\usepackage{booktabs}
\usepackage{siunitx}
\usepackage{pdfpages}
\usepackage{fontspec}
\usepackage{hyperref}

\setmainfont{Cambria}
\setsansfont{Segoe UI}
\setmonofont{Consolas}

\begin{document}
\begin{titlepage}
    \begin{center}
        \vspace*{1cm}

        \textbf{\Large{LAB 2}}

        \vspace{0.5cm}

        \LARGE{A Multithreaded Server to Handle Concurrent Read and Write Requests}

        \vspace{1.5cm}

        \textbf{\Large{Cynthia Li, Michael Kwok, Samuel Lojpur}}

        \vfill

        ECE 420 -- Parallel and Distributed Programming\\
        Department of Electrical and Computer Engineering\\
        University of Alberta\\
        \today

    \end{center}
\end{titlepage}

\tableofcontents
\listoffigures
\listoftables

\pagebreak

\section{Description of Implementation}


Describe your implementation. It should be detailed enough so that other people can repeat your work; however, avoid including raw code in your description

\section{Performance Discussion}

Discuss some particular performance issues as required in each lab, and support them w/ your experimental results. Typical topics include, but are not limited to, speedup, efficiency, running time as a function of the number of processes and problem size. You can also discuss additional points related to performance that are not required in the lab manual (overhead, cost, etc). Use figures/tables to show your results.

- table comparing different schemes over different array sizes
- compare mean/median processing times of diff server implementations (avg over 100 runs) in terms of average memory access latency of the 1000 client requests
- repeat experiment w/ different n, the size (number of strings) of theArray
- n should take values 10, 100, 1000
- explain observations

- performance interpretations

\subsection{Performance Table}

\begin{table}[H]
\centering
\begin{subfigure}{\linewidth}
    \centering
    \begin{tabular}{c c c | c c}
        \toprule
        $A$ & $B$ & $C_{in}$ & $Sum$ & $C_{out}$ \\
        \midrule
        0   & 0   & 0        & 0     & 0         \\
        0   & 0   & 1        & 1     & 0         \\
        0   & 1   & 0        & 1     & 0         \\
        0   & 1   & 1        & 0     & 1         \\
        1   & 0   & 0        & 1     & 0         \\
        1   & 0   & 1        & 0     & 1         \\
        1   & 1   & 0        & 0     & 1         \\
        1   & 1   & 1        & 1     & 1         \\
        \bottomrule
    \end{tabular}
    \caption{Truth Table for 1-bit full adder}
    \label{tab:truth-table-1b}
\end{subfigure}

\caption{Truth tables for 1-bit full adder}
\end{table}

Initially, a truth table for a 1-bit full adder was made (Table~\ref{tab:truth-table-1b}).
Corresponding Karnaugh maps were subsequently constructed from the values (Figures~\ref{diag:kmap-f} \&~\ref{diag:kmap-c}).
As can be seen, $F$ can be constructed by simply getting the xor of the terms, $F = A \oplus B \oplus C$,
and $C_{out}$ can be constructed by minterm implicants: $C_{out} = AB + BC_{in} + AC_{in}$.
For a 2-bit adder, the previous expressions can be applied to each bit to form each corresponding output bit,
with the $C_{out}$ of the least significant bit connected to the $C_{in}$ of the most significant bit.

\begin{figure}[H]
    \centering
    \includegraphics[width=0.53\textwidth]{rtl-schematic.png}
    \caption{Block Diagram of RTL design}
    \label{fig:rtl-schematic}
\end{figure}

This circuit was implemented at the RTL level in VHDL, as shown in the code in the appendix.
Two instances of the adder were chained together to form a ripple-carry 2-bit full adder,
which produced a circuit with three inputs and two outputs: two 2-bit vectors for the summands,
a 1-bit input for the carry-in, a 1-bit carry out and a 2-bit vector sum output.
The synthesized block diagram is shown in Figure~\ref{fig:rtl-schematic}.

\subsection{MUX Design}

A second version of the circuit was created, using three 16x1 MUXes to try out MUX-based circuit design.
Each MUX used the concatenated values of A and B as the selector.
The output was based on either short circuit to ground, VDD, $C_{in}$ or $\overline{C_{in}}$.

\begin{table}[H]
    \centering
    \begin{subfigure}{0.4\linewidth}
        \caption{Full table}
        \label{tab:truth-table-mux}
        \begin{tabular}{c c c | c c c}
            \toprule
            $A$ & $B$ & $C_{in}$ & $C_{out}$ & $S_1$ & $S_0$ \\
            \midrule
            00  & 00  & 0        & 0         & 0     & 0     \\
            00  & 00  & 1        & 0         & 0     & 1     \\
            00  & 01  & 0        & 0         & 0     & 1     \\
            00  & 01  & 1        & 0         & 1     & 0     \\
            00  & 10  & 0        & 0         & 1     & 0     \\
            00  & 10  & 1        & 0         & 1     & 1     \\
            00  & 11  & 0        & 0         & 1     & 1     \\
            00  & 11  & 1        & 1         & 0     & 0     \\
            01  & 00  & 0        & 0         & 0     & 1     \\
            01  & 00  & 1        & 0         & 1     & 0     \\
            01  & 01  & 0        & 0         & 1     & 0     \\
            01  & 01  & 1        & 0         & 1     & 1     \\
            01  & 10  & 0        & 0         & 1     & 1     \\
            01  & 10  & 1        & 1         & 0     & 0     \\
            01  & 11  & 0        & 1         & 0     & 0     \\
            01  & 11  & 1        & 1         & 0     & 1     \\
            10  & 00  & 0        & 0         & 1     & 0     \\
            10  & 00  & 1        & 0         & 1     & 1     \\
            10  & 01  & 0        & 0         & 1     & 1     \\
            10  & 01  & 1        & 1         & 0     & 0     \\
            10  & 10  & 0        & 1         & 0     & 0     \\
            10  & 10  & 1        & 1         & 0     & 1     \\
            10  & 11  & 0        & 1         & 0     & 1     \\
            10  & 11  & 1        & 1         & 1     & 0     \\
            11  & 00  & 0        & 0         & 1     & 1     \\
            11  & 00  & 1        & 1         & 0     & 0     \\
            11  & 01  & 0        & 1         & 0     & 0     \\
            11  & 01  & 1        & 1         & 0     & 1     \\
            11  & 10  & 0        & 1         & 0     & 1     \\
            11  & 10  & 1        & 1         & 1     & 0     \\
            11  & 11  & 0        & 1         & 1     & 0     \\
            11  & 11  & 1        & 1         & 1     & 1     \\
            \bottomrule
        \end{tabular}
    \end{subfigure}
    \begin{subfigure}{0.4\linewidth}
        \caption{Simplified for MUX}
        \label{tab:mux-simplified}
        \begin{tabular}{c c | c c c}
            \toprule
            $A$ & $B$ & $C_{out}$ & $S_1$               & $S_0$               \\
            \midrule
            00  & 00  & 0         & 0                   & $C_{in}$            \\
            00  & 01  & 0         & $C_{in}$            & $\overline{C_{in}}$ \\
            00  & 10  & 0         & 1                   & $C_{in}$            \\
            00  & 11  & $C_{in}$  & $\overline{C_{in}}$ & $\overline{C_{in}}$ \\
            01  & 00  & 0         & $C_{in}$            & $\overline{C_{in}}$ \\
            01  & 01  & 0         & 1                   & $C_{in}$            \\
            01  & 10  & $C_{in}$  & $\overline{C_{in}}$ & $\overline{C_{in}}$ \\
            01  & 11  & 0         & 0                   & $C_{in}$            \\
            10  & 00  & 0         & 1                   & $C_{in}$            \\
            10  & 01  & $C_{in}$  & $\overline{C_{in}}$ & $\overline{C_{in}}$ \\
            10  & 10  & 0         & 0                   & $C_{in}$            \\
            10  & 11  & 0         & $C_{in}$            & $\overline{C_{in}}$ \\
            11  & 00  & $C_{in}$  & $\overline{C_{in}}$ & $\overline{C_{in}}$ \\
            11  & 01  & 0         & 0                   & $C_{in}$            \\
            11  & 10  & 0         & $C_{in}$            & $\overline{C_{in}}$ \\
            11  & 11  & 0         & 1                   & $C_{in}$            \\
            \bottomrule
        \end{tabular}
    \end{subfigure}
    \caption{Truth tables for 2-bit full adder}
\end{table}

\begin{figure}[h]
    \centering
    \includegraphics[width=0.5\linewidth]{mux-schematic.png}
    \caption{Block Diagram of MUX design}
    \label{fig:mux-schematic}
\end{figure}

As part of this process, a 32 entry truth table was constructed (Table~\ref{tab:truth-table-mux}).
This was extremely time consuming and error prone, as the work was ``monotonous'', and it quickly became hard to keep track
of the binary numbers that were being written. While a Karnaugh Map was not required,
the values for $C_{out}$, $S_1$ and $S_0$ had to be simplified by hand to and redefined in terms of $C_{in}$ for each MUX input.
The result of simplification is shown in Table~\ref{tab:mux-simplified}.
This all resulted in the synthesis of 3 MUXes for the circuit (Figure~\ref{fig:mux-schematic}), as expected.

\subsection{Comparator}

For the comparator, a behavioural implementation was done. This was to allow the synthesis and implementation tools to
optimize the design as best as possible by providing it with semantically as much information as possible.

The schematic generated by the tool can be seen in Figure~\ref{fig:comp-schematic}.

\begin{figure}[h]
    \centering
    \includegraphics[width=0.6\linewidth]{comp-schematic.png}
    \caption{Block Diagram of Comparator design}
    \label{fig:comp-schematic}
\end{figure}

\pagebreak

\subsection{FPGA Implementation}

In terms of implementation, the MUX design produced a less efficient circuit resource utilization-wise,
with 3 LUTs required to implement the adder as compared to 2 for RTL.\@
A possible explanation for this could be that each 1-bit full adder was implemented as a LUT for the RTL Design,
instead of a single LUT for each output bit for the MUX Design.

\begin{figure}[h]
    \centering
    \begin{subfigure}{0.45\linewidth}
        \centering
        \includegraphics[width=\textwidth]{mux-resource.png}
        \caption{MUX Design}
    \end{subfigure}
    \begin{subfigure}{0.45\linewidth}
        \centering
        \includegraphics[width=\textwidth]{rtl-resource.png}
        \caption{RTL Design}
    \end{subfigure}
    \caption{Resource utilization of both designs}
    \label{fig:resource}
\end{figure}

The MUX design might be faster than the RTL design since there is no need to wait for the result of the addition of the first bit,
and the entire 2-bit output is produced at the same time. In order to prove this, timing analysis must be done,
which is unnecessary for a 2-bit adder as any delay due to ripple carry is likely to be below the margin of error.

\section{Testing \& Simulation}

\begin{figure}[h]
    \centering
    \includegraphics[width=\linewidth]{tb-result.png}
    \caption{Waveform Screenshot}
    \label{fig:tb-result}
\end{figure}

For the testing and simulation of the components, a testbench was written and ran for both implementations of the device.
The resulting waveforms were compared to ensure proper operation of the design, and the waveform can be seen in Figure~\ref{fig:tb-result}.

The designed testbench was written to loop through all possible values of $A$, $B$ and $C_{in}$.
As shown, the adders worked as expected.

The designs were tested on the Zybo board provided, and the expected results were obtained by manipulating the buttons and switches on the board.

% \section{References}

\pagebreak
\section*{Appendix}

\subsection{Top Level File}
\inputminted{vhdl}{../src/lab1_top.vhd}

\pagebreak
\subsection{1-bit full adder}
\inputminted{vhdl}{../src/full_adder_1bit.vhd}

\pagebreak
\subsection{2-bit full adder}
\inputminted{vhdl}{../src/full_adder_2bit.vhd}

\pagebreak
\subsection{Cout MUX}
\inputminted{vhdl}{../src/Co_mux.vhd}

\pagebreak
\subsection{S1 MUX}
\inputminted{vhdl}{../src/S1_mux.vhd}

\pagebreak
\subsection{S0 MUX}
\inputminted{vhdl}{../src/S0_mux.vhd}

\pagebreak
\subsection{Comparator}
\inputminted{vhdl}{../src/Comparator.vhd}

\pagebreak
\subsection{Testbench}
\inputminted{vhdl}{../test/full_adder_2bit_tb.vhd}

\renewcommand{\thepage}{}

\end{document}
