\RequirePackage[l2tabu, orthodox]{nag}

\documentclass{article}

\usepackage[margin=1.9cm, letterpaper]{geometry}
\usepackage[utf8]{inputenc}
\usepackage[a-1b]{pdfx}
\usepackage{indentfirst}
\usepackage{float}
\usepackage{subcaption}
\usepackage{mathtools}
\usepackage{amssymb}
\usepackage{booktabs}
\usepackage{pdfpages}
\usepackage{fontspec}
\usepackage{hyperref}

\setmainfont{Cambria}
\setsansfont{Segoe UI}
\setmonofont{Consolas}

\begin{document}
\begin{titlepage}
    \begin{center}
        \vspace*{1cm}

        \textbf{\Large{LAB 2}}

        \vspace{0.5cm}

        \LARGE{A Multithreaded Server to Handle Concurrent Read and Write Requests}

        \vspace{1.5cm}

        \textbf{\Large{Cynthia Li, Michael Kwok, Samuel Lojpur}}

        \vfill

        ECE 420 -- Parallel and Distributed Programming\\
        Department of Electrical and Computer Engineering\\
        University of Alberta\\
        \today

    \end{center}
\end{titlepage}

\tableofcontents
\listoffigures
\listoftables

\pagebreak

\section{Description of Implementation}

% Describe your implementation. It should be detailed enough so that other people can repeat your work; however, avoid including raw code in your description

\section{Performance Discussion}

% Discuss some particular performance issues as required in each lab, and support them w/ your experimental results. Typical topics include, but are not limited to, speedup, efficiency, running time as a function of the number of processes and problem size. You can also discuss additional points related to performance that are not required in the lab manual (overhead, cost, etc). Use figures/tables to show your results.

% - table comparing different schemes over different array sizes
% - compare mean/median processing times of diff server implementations (avg over 100 runs) in terms of average memory access latency of the 1000 client requests
% - repeat experiment w/ different n, the size (number of strings) of theArray
% - n should take values 10, 100, 1000
% - explain observations

% - performance interpretations

% \begin{table}[H]
%     \centering
%     \begin{subfigure}{\linewidth}
%         \centering
%         \begin{tabular}{c c c | c c}
%             \toprule
%             $A$ & $B$ & $C_{in}$ & $Sum$ & $C_{out}$ \\
%             \midrule
%             0   & 0   & 0        & 0     & 0         \\
%             0   & 0   & 1        & 1     & 0         \\
%             0   & 1   & 0        & 1     & 0         \\
%             0   & 1   & 1        & 0     & 1         \\
%             1   & 0   & 0        & 1     & 0         \\
%             1   & 0   & 1        & 0     & 1         \\
%             1   & 1   & 0        & 0     & 1         \\
%             1   & 1   & 1        & 1     & 1         \\
%             \bottomrule
%         \end{tabular}
%         \caption{Table 1}
%         \label{tab:table-1}
%     \end{subfigure}

%     \caption{}
% \end{table}

% \begin{figure}[H]
%     \centering
%     \includegraphics[width=0.53\textwidth]{rtl-schematic.png}
%     \caption{Block Diagram of RTL design}
%     \label{fig:rtl-schematic}
% \end{figure}


\pagebreak
\section*{Appendix}

\renewcommand{\thepage}{}

\end{document}
